\section*{Problema 3 (CAS)}
Completa la tabla de Diff. En particular, explique por qué los valores grandes de $h$ dan la misma aproximación para la hora de la muerte.

\begin{center}
    Con la ecuación (4), $T(t) = 98$ y $t > h$\\ 
    \vspace{10pt}
    \begin{tabular}{ |c|c|c|c| } 
        \hline
        $h$ & $t$ & Hora en que movieron el cuerpo & Hora de la muerte \\ 
        \hline
        12 & ~5.448081 & 18:00 &\\
        \hline
        11 & ~5.448081 & 19:00 &\\
        \hline
        10 & ~5.448081 & 20:00 &\\
        \hline
        9 & ~5.448081 & 21:00 &\\
        \hline
        8 & ~5.448081 & 22:00 &\\
        \hline
        7 & ~5.448081 & 23:00 &\\
        \hline
        6 & ~5.448081 & 0:00 &\\
        \hline
        5 & ~5.775420 & 1:00 &\\
        \hline
        4 & ~6.562310 & 2:00 &\\
        \hline
        3 & ~7.437305 & 3:00 &\\
        \hline
        2 & ~8.415677 & 4:00 &\\
        \hline
    \end{tabular}
\end{center}
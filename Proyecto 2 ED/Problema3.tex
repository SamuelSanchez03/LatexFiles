\section*{Problema 3 (CAS)}
Completa la tabla de Diff. En particular, explique por qué los valores grandes de $h$ dan la misma aproximación para la hora de la muerte.

\begin{center}
    Con la ecuación (4), $T(t) = 99$ y $t > h$\\ 
    \vspace{10pt}
    \begin{tabular}{ |c|c|c|c| } 
        \hline
        $h$ & $t$ & Hora en que movieron el cuerpo & Hora de la muerte \\ 
        \hline
        12 & ~2.5425 & 18:00 & 15:27\\
        \hline
        11 & ~2.9578 & 19:00 & 16:03\\
        \hline
        10 & ~3.4091 & 20:00 & 16:36\\
        \hline
        9 & ~3.9006 & 21:00 & 17:06\\
        \hline
        8 & ~4.4371 & 22:00 & 17:34\\
        \hline
        7 & ~5.0246 & 23:00 & 17:59\\
        \hline
        6 & ~5.6697 & 0:00 & 18:20\\
        \hline
        5 & ~6.3807 & 1:00 & 18:37\\
        \hline
        4 & ~7.1676 & 2:00 & 18:50\\
        \hline
        3 & ~8.0426 & 3:00 & 18:57\\
        \hline
        2 & ~9.0210 & 4:00 & 18:59\\
        \hline
    \end{tabular}

    \vspace{10pt}

    \begin{tikzpicture}
    \begin{axis}[xmax = 15, xmin = -15, ymax = 100, ymin = 0, axis x line = bottom, axis y line=middle, legend pos=outer north east, xlabel = $h$, ylabel = $t$]
    \addplot[color = red, samples=1000, domain=-15:20]{ln(29/(35-20*exp(2*ln(34/35)*x)))/(-2*ln(34/35))};
    %\addlegendentry{\(Ecuación (4)\)}
    \end{axis}
    \end{tikzpicture}

    Gráfica Ecuación (4)\\
    La Ecuación (4) al despejarla para que $t$ dependa de $h$, tenemos:
    \begin{equation*}
        t = \frac{\ln{\left(\frac{29}{35-20e^{-kh}}\right)}}{k}, k = -2\ln{\left(\frac{34}{35}\right)}
    \end{equation*}
\end{center} 
Podemos observar que se trata de una función exponencial inversa por lo que cuando crece $h$, $t$ se hace cada vez más pequeño, por lo que da una hora de muerte cada vez más cercana a la hora en la que se movió el cuerpo. Por otro lado, cuando $h$ decrece, $t$ se hace cada vez más grande, lo que causa que la hora de muerte se aleje cada vez más de la hora en la que se movió el cuerpo y tienda a un sólo valor, aproximándose a las 19 hrs. como hora de la muerte.
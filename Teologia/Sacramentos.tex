\documentclass{article}

\usepackage[english, spanish]{babel}


\selectlanguage{spanish}
 
\title{Los sacramentos}

\begin{document}
    \maketitle
    \section*{Signos que producen vida}
        \begin{itemize}
            \item La redención que nos consiguió la muerte de Jesucristo debía aplicarse a cada hombre singular, y para ello dejó establecidos los cauces. 
            \item Esos cauces son los siete sacramentos.
            \item Nos comunican la gracia, que es una realidad invisible, a través de algo sensible, material, tal como corresponde a nuestra forma de ser, corpórea y espiritual a un tiempo.
            \item Para que haya sacramento hace falta alguna realidad visible de este mundo, algo tangible. Jesús escogió algo para hacer los sacramentos.
            \item Él manda su gracia invisible a través de cosas visibles, comunes. Nos libra del peligro de no saber cuándo ni cómo recibiríamos aquello que no podemos ver, ni sentir, ni tocar.
            \item El sacramento también es el lugar de expresión máxima de una realidad humana que tiene ya de por sí un significado.
            \item El sacramento porta un mensaje. Los sacramentos sí contienen el Amor que Dios nos profesa. 
            \item Los sacramentos profesan la gracia \textit{ex opere operato}, por la obra realizada. 
            \item Los sacramentos no son acciones humanas sino acciones divinas. 
        \end{itemize}

    \section*{La gracia es la vida del alma}
       'Gracia' designa todo don de Dios.\\
       Designa los dones a los que el hombre no tiene derecho ni siquiera remotamente, a los que su naturaleza humana no le dan acceso.\\  
       Dones que estan sobre la naturaleza humana. Don sobre-natural de Dios.\\
       'Gracia' es aquello que está por encima de cualquier naturaleza creada o creable, es algo exclusivo y propio de Dios. Es el don sobrenatural por el que una criatura racional 'participa de la misma vida íntima de Dios'.\\
       Por la gracia santificante gozamos de la Vida de Dios, somos sus hijos. Si morimos en ese estado, por esa gracia Dios hará que vivamos eternamente en su intimidad gozosa. Sin ella nuestra alma está vacía y oscura, muerta sobrenaturalmente.\\
       La gracia santificante es distinta de Dios, pero fluye de Él y es resultado de Su Presencia en el alma.
       Presencia de inhabitación, de esta segunda manera no puede Dios habitar en las cosas materiales, pero allá donde haya un espíritu podrá descender y conversar con él.
       Se produce cuando Dios infunde su gracia a ese espíritu, y entonces lo hace apto para que las Personas divinas establezcan ahí su morada.\\

       Dos clases de amor:
       \begin{itemize}
       \item Amor común. Se extiende a todo lo que existe en tanto que existe.
       \item Amor especial. Que tiene al ser espiritual que ha recibido la gracia. Eleva a la criatura espiritual sobre las condiciones de su naturaleza, revistiéndola de una nueva, introduciéndola en un nuevo universo. 
       \end{itemize}

       Con la gracia santificante nuestras acciones adquieren mérito, valen ante Dios. No existe mérito sobrenatural en las acciones de quien está en pecado mortal.
       Por mérito se entiende aquella propiedad de una obra buena que otorga al que la realiza el derecho a un premio. 
    \section*{La gracia que impulsa}
       Cuando un hombre pierde la gracia santificante al cometer pecado mortal, ha destruido el espléndido brillo de su alma, ha extinguido en ella la vida sobrenatural. Sólo la gracia puede devolver el brillo al alma.\\ 
       La gracia santificante nos devuelve al mundo de los vivos, pero para ser capaces de recobrar esa misma gracia, necesitamos otro tipo de ayudas: \textit{gracias actuales}.\\ 
       Las gracias actuales son un regalo continuo de Dios. Aquellas mociones divinas que iluminan nuestro entendimiento o mueven y confortan a la voluntad, para hacernos capaces de realizar acciones sobrenaturales. La gracia actual la requerimos todos.
    \section*{Algo más sobre los sacramentos}
       \begin{itemize}
            \item \textit{Signos sensibles}. Nuestros sentidos los pueden percibir. 
            \begin{itemize}
                \item Materia. Es la cosa que se utiliza.
                \item Forma. Palabras que den sentido a la acción del sacramento. 
            \end{itemize}
            \item \textit{Instituidos por cristo}. En el intervalo transcurrido entre el inicio de su vida pública y marcha al cielo, Jesús instituyó los siete sacramentos. No hay poder humano capaz de relacionar la gracia sobrenatural con un signo material.  
            \item \textit{Eficaz de la gracia}. Su propósito esencial es infundir la gracia. Si no la infundieran por sí mismos, estos signos sensibles no serían sacramentos, aunque hubieran tenido su origen en la vida del Señor.
       \end{itemize}
       Si la gracia santificante fuera la única ayuda que Dios buscara darnos, bastaría un sólo sacramento. Pero Dios provee de ayuda adecuada a las necesidades de nuestro estado personal en la vida. Esta ayuda es la 'gracia sacramental'.\\
       La gracia sacramental no es sino la gracia actual que Dios se compromete a dar para ayudar a alguien en algo específico. 
    \section*{Acciones de Cristo}
       Los sacramentales obtienen su eficacia principalmente por las oraciones que la Iglesia asocia por aquellos que usen ese sacramental. La plegaria que la Iglesia ya realizó ores lo que lo hace vehículo de la gracia. El signo externo de un sacramental por sí en sí no tiene la facultad para perdonar los pecados. En cambio, si nos acercamos al sacramento de la confesión con las debidas dispociones, el sacramento - por sí y en sí - nos perdona todos los pecados. 
    \section*{Tres sacramentos imprimen el carácter}

\end{document}
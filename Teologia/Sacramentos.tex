\documentclass[letterpaper, 12pt]{article}
\usepackage{array}
\usepackage{gensymb}
\usepackage{tikz, pgfplots }
\usepackage[english, spanish]{babel}
\usetikzlibrary{shapes, arrows, positioning, plotmarks}

\selectlanguage{spanish}
 
\title{Los sacramentos}

\begin{document}
    \maketitle
    \section*{Signos que producen vida}
        \begin{itemize}
            \item La redención que nos consiguió la muerte de Jesucristo debía aplicarse a cada hombre singular, y para ello dejó establecidos los cauces. 
            \item Esos cauces son los siete sacramentos.
            \item Nos comunican la gracia, que es una realidad invisible, a través de algo sensible, material, tal como corresponde a nuestra forma de ser, corpórea y espiritual a un tiempo.
            \item Para que haya sacramento hace falta alguna realidad visible de este mundo, algo tangible. Jesús escogió algo para hacer los sacramentos.
            \item Él manda su gracia invisible a través de cosas visibles, comunes. Nos libra del peligro de no saber cuándo ni cómo recibiríamos aquello que no podemos ver, ni sentir, ni tocar.
            \item El sacramento también es el lugar de expresión máxima de una realidad humana que tiene ya de por sí un significado.
            \item El sacramento porta un mensaje. Los sacramentos sí contienen el Amor que Dios nos profesa. 
            \item Los sacramentos profesan la gracia \textit{ex opere operato}, por la obra realizada. 
            \item Los sacramentos no son acciones humanas sino acciones divinas. 
        \end{itemize}

    \section*{La gracia es la vida del alma}
    
    \section*{La gracia que impulsa}

    \section*{Algo más sobre los sacramentos}

    \section*{Acciones de Cristo}

    \section*{Tres sacramentos imprimen el carácter}
    
\end{document}